%%%%%%%%%%%%%%%%%%%%%%%%%%%%%%%%%%%%%%%%%%%%%%%%%%%%%%%%%%%%%%%%%
% IMPORTANTE: PARA VER EL MISMO CÓDIGO MÁS LIMPIO Y SIN COMENTARIOS, ABRIR EL ARCHIVO "main.tex", CONTIENE EL MISMO CÓDIGO PERO SIN COMENTARIOS EXPLICATIVOS. LA ESTRUCTURA Y TÍTULOS DE CAPÍTULOS Y SUBCAPÍTULOS SON REFERENCIALES, MODIFICAR DE ACUERDO A SU NECESIDAD.
% LOS ARCHIVOS QUE DEBEN EVITAR SER ELIMINADOS O MODIFICADOS SON preambulo.tex, bibliografia.tex, bibliografia.bib, 0_portada.tex, 3_indices.tex (este puede ser modificado de nombre, pero no de contenido). Los demás archivos, son modificables, eliminables y se espera que cada quien coloque sus propios archivos.tex.
%%%%%%%%%%%%%%%%%%%%%%%%%%%%%%%%%%%%%%%%%%%%%%%%%%%%%%%%%%%%%%%%%
%%%%%%%%%%%%%%%%%%%%%%%%%%%%%%%%%%%%%%%%%%%%%%%%%%%%%%%%%%%%%%%%%
% DOCUMENTO DE TIPO LIBRO, TAMAÑO A4 Y LETRA DE 12pt: Tipo {book}, Tamaño de hoja {A4}, Tamaño de letra {12pt}, eliminación de páginas en blanco con {oneside}, si se desea imprimir en doble cara y asegurar el formato, borrar configuración "oneside" (eso aumentaría páginas en blanco).
\documentclass[a4paper, 12pt, openany, oneside]{book}
%%%%%%%%%%%%%%%%%%%%%%%%%%%%%%%%%%%%%%%%%%%%%%%%%%%%%%%%%%%%%%%%%
% IMPORTANTE, CONFIGURACIONES Y PAQUETES PRINCIPALES: "preámbulo.tex" contiene la definición del documento: se incluyen paquetes para la compatibilidad de diferentes caracteres, configuraciones de márgenes {2.54cm}, configuraciones de descripciones o pie de figuras y tablas para que se adecúen al formato APA (alineados hacia la izquierda, en negrita y cursiva, revisar formato APA), inclusión de paquetes para matemáticas, e inclusión de código, configuración de bibliografía con biblatex y se añade al archivo "bibliografia.bib" como fuente bibligráfica.
%%%%%%%%%%%%%%%%%%%%%%%%%%%%%%%%%%%%%%%%%%%%%%%%%%%%%%%%%%%%%%%
% PAQUETES PRINCIPALES %%%%%%%%%%%%%%%%%%%%%%%%%%%%%%%%%%%%%%%%
\usepackage[utf8]{inputenc} % Evita problemas con caracteres especiales
\usepackage[T1]{fontenc} % Permite manejo caracteres acentuados y caracteres especiales
\usepackage[spanish]{babel} % Gestiona la traducción de elementos del archivo
\usepackage{tocbibind} % Agrega lista de figuras, tablas y bibliografía al índice
\usepackage{xcolor} % Para comando \textcolor{color}{text}
\usepackage{float}
%%%%%%%%%%%%%%%%%%%%%%%%%%%%%%%%%%%%%%%%%%%%%%%%%%%%%%%%%%%%%%%
% PAQUETE PARA MANEJO E INCLUSIÓN DE IMÁGENES %%%%%%%%%%%%%%%%%
\usepackage{graphicx}
%%%%%%%%%%%%%%%%%%%%%%%%%%%%%%%%%%%%%%%%%%%%%%%%%%%%%%%%%%%%%%%
% PAQUETES DE CONFIGURACIÓN DE MÁRGENES E INTERLINEADO %%%%%%%%
\usepackage{geometry} % Para personalización de márgenes
\geometry{
	left=2.54cm,
	right=2.54cm,
	top=2.54cm,
	bottom=2.54cm
}
\usepackage{setspace} % Gestión de interlineado, opciones: \singlespacing, \onehalfspacing o\doublespacing.
%%%%%%%%%%%%%%%%%%%%%%%%%%%%%%%%%%%%%%%%%%%%%%%%%%%%%%%%%%%%%%%
% CONFIGURACIÓN DE PÁGINAS %%%%%%%%%%%%%%%%%%%%%%%%%%%%%%%%%%%%
\usepackage{pdflscape} % Configuración de página para manejar gráficos o tablas amplias (para girar las páginas horizontalmente)
%%%%%%%%%%%%%%%%%%%%%%%%%%%%%%%%%%%%%%%%%%%%%%%%%%%%%%%%%%%%%%%
% PAQUETES Y CONFIGURACIONES DE TÍTULOS, PIE DE FIGURAS Y TABLAS
\usepackage{titlesec} % Permite la personalización de títulos y secciones
\usepackage{caption} % Permite la personalización de títulos de figuras y tablas
\usepackage{subcaption} % Permite la inclusión de subtítulos en una figura principal
% Si no se tienen tablas ni figuras aún, los comandos \captionsetuo darán mensaje de advertencia, los cuales desaparecerán cuando se coloquen figuras y tablas en el cuerpo
\captionsetup[figure]{justification=RaggedRight,labelsep=newline,labelfont=bf, textfont=it, position=top,singlelinecheck=false}
\captionsetup[table]{justification=RaggedRight,labelsep=newline,labelfont=bf,textfont=it,position=top,singlelinecheck=false}
% Configuración de índice de Tablas
\addto\captionsspanish{\renewcommand{\listtablename}{Índice de Tablas}}
% Cómo se describiran las tablas: Tabla, Cuadro, etc. Modificar de acuerdo a su necesidad
\addto\captionsspanish{\renewcommand{\tablename}{Tabla}}
\usepackage{hyperref} % Manejo de hiperreferencias
%%%%%%%%%%%%%%%%%%%%%%%%%%%%%%%%%%%%%%%%%%%%%%%%%%%%%%%%%%%%%%%
% PAQUETES PARA MANEJO DE TABLAS %%%%%%%%%%%%%%%%%%%%%%%%%%%%%%
\usepackage{array} % Permite la gestión de tablas
\usepackage{multirow} % Permite la configuración de multifila
\usepackage{threeparttable} % Tablas con notas
%%%%%%%%%%%%%%%%%%%%%%%%%%%%%%%%%%%%%%%%%%%%%%%%%%%%%%%%%%%%%%%
% PAQUETE DE RENDERIZACIÓN DE CAJAS PARA TEXTO %%%%%%%%%%%%%%%%
\usepackage{tcolorbox} % Cajas de texto personalizadas
%%%%%%%%%%%%%%%%%%%%%%%%%%%%%%%%%%%%%%%%%%%%%%%%%%%%%%%%%%%%%%%
% PAQUETES PARA USO DE MATEMÁTICAS %%%%%%%%%%%%%%%%%%%%%%%%%%%%
\usepackage{amsmath, amssymb}
\usepackage{newtxtext,newtxmath} % Modifica la fuente a una parecida a Times New Roman
%%%%%%%%%%%%%%%%%%%%%%%%%%%%%%%%%%%%%%%%%%%%%%%%%%%%%%%%%%%%%%%
% CONFIGURACIÓN PARA INCUSIÓN DE CÓDIGOS EN EL TEXTO %%%%%%%%%%
\usepackage{listings} % Para código fuente
\usepackage{inconsolata} % Fuente monoespaciada
\lstset{
	language=Python,
	backgroundcolor=\color{white},
	basicstyle=\footnotesize\ttfamily,
	keywordstyle=\color{blue}\bfseries,
	stringstyle=\color{red},
	commentstyle=\color{green!50!black},
	frame=single,
	breaklines=true,
	numbers=left,
	numberstyle=\tiny\color{gray},
	tabsize=4
}
%%%%%%%%%%%%%%%%%%%%%%%%%%%%%%%%%%%%%%%%%%%%%%%%%%%%%%%%%%%%%%%
% CONFIGURACIÓN Y GESTIÓN DE BIBLIOGRAFÍA CON COMPILADOR BIBLATEX
\usepackage[style=apa, backend=biber]{biblatex}
\usepackage[babel]{csquotes} % Manejo avanzado de citas
% Mapeo de bibliografía estilo APA en español
\DeclareLanguageMapping{spanish}{spanish-apa}
\DefineBibliographyExtras{spanish}{
	\renewcommand*{\finalnamedelim}{\addspace y\space}
	\renewcommand*{\multinamedelim}{\addcomma\space}
	\renewcommand*{\multilistdelim}{\addcomma\space}
}
\makeatletter
\DefineBibliographyExtras{spanish}
{
	\setcounter{smartand}{1}
	\let\lbx@finalnamedelim=\lbx@es@smartand
	\let\lbx@finallistdelim=\lbx@es@smartand
}
\makeatother  
% Añadir archivo de bibliografía
\addbibresource{bibliografia.bib}
%%%%%%%%%%%%%%%%%%%%%%%%%%%%%%%%%%%%%%%%%%%%%%%%%%%%%%%%%%%%%%%
% Numeración de páginas %%%%%%%%%%%%%%%%%%%%%%%%%%%%%%%%%%%%%%%
\pagestyle{plain}
%%%%%%%%%%%%%%%%%%%%%%%%%%%%%%%%%%%%%%%%%%%%%%%%%%%%%%%%%%%%%%%%%
% INICIO DEL DOCUMENTO Y RENDERIZACIÓN PDF %%%%%%%%%%%%%%%%%%%%%%
%%%%%%%%%%%%%%%%%%%%%%%%%%%%%%%%%%%%%%%%%%%%%%%%%%%%%%%%%%%%%%%%%
\begin{document}
    % IMPORTANTE, CONFIGURACIONES DE PORTADA: En el archivo "0_portada.tex", modificar los datos personales e institucionales necesarios y, si es preciso, modificar el logo del escudo en "figuras/logo.png".
    \begin{titlepage}
   \begin{center}
       
       {\large \textbf{UNIVERSIDAD NACIONAL DE SAN ANTONIO ABAD DEL CUSCO}}\\
       \vspace{0.5cm}
       % MOFIFICAR SI FUERA NECESARIO
       {\normalsize\textbf{FACULTAD DE INGENIERÍA ELÉCTRICA, ELECTRÓNICA, INFORMÁTICA Y MECÁNICA}}\\
       \vspace{0.5cm}
       % MODIFICAR SI FUERA NECESARIO
       {\normalsize\textbf{ESCUELA PROFESIONAL DE XXXXXX XXXXXX}}\\
       
       \vspace{1cm}
       
       \includegraphics[scale=0.26]{figuras/logo.png}\\
       
       \vspace{1cm}
       
       \textbf{TESIS}
       
       \vspace{0.3cm}
       
       \tcolorbox[
           colframe=black,
           colback=white,
           arc=3mm,
           boxrule=1pt,
           width=1\textwidth,
           halign=center,
           before=\raggedright
       ]
       \begin{center}
           % MODIFICAR NOMBRE DE TESIS
           {\textbf{XXXXXX XXXXXX XXXXXX XXXXXX XXXXXX XXXXXX XXXXXX XXXXXX XXXXXX XXXXXX XXXXXX} \par}
       \end{center}
       \endtcolorbox
       
       \vspace{1cm}
       
       \begin{flushright}
           \begin{minipage}{0.53\textwidth} 
               \textbf{PRESENTADO POR:}\\
               % MODIFICAR NOMBRE DEL TESISTA
               Br. XXXXXX XXXXXX XXXXXX XXXXXX
               
               \vspace{0.5cm}

               % MODIFICAR TÍTULO A OBTENER
               \textbf{PARA OPTAR AL TÍTULO PROFESIONAL DE XXXXXX XXXXXX}
               
               \vspace{0.5cm}
               
               \textbf{ASESOR:}\\
               % MODIFICAR ASESOR
               Dr. XXXXXX XXXXXX XXXXXX
           \end{minipage}
       \end{flushright}
       
       \vspace{0.5cm}

       \begin{center}
            \textbf{CUSCO - PERÚ}
            \vspace{0.5cm}
            
            \textbf{\the\year}
       \end{center}
   \end{center}
\end{titlepage}

    % FRONTMATTER: cambia a numeración romana para colocar Dedicatoria, Agradecimientos, Índices, Resumen (abstract) e Introducción (modificable de acuerdo a su interés, puede crear su respectivo archivo "xxx.tex" e incluirlo con \input{xxx.tex}). El archivo "3_indices.tex", tiene la configuración de importación de los índices, cuidado al modificarlo o eliminarlo, puede ser colocado en otro orden y cambiarle el nombre si es necesario, pero evitar modificar su contenido.
    \frontmatter
    \chapter*{Dedicatoria}
\doublespacing % Espaciado doble, puede configurarse con: \singlespacing (1.0), \onehalfspacing (1.5) o\doublespacing (2.0).

A mi familia, por su incondicional apoyo\\
y todas las bondades que me brinda.
    \chapter*{Agradecimiento}
\doublespacing % Espaciado doble, puede configurarse con: \singlespacing (1.0), \onehalfspacing (1.5) o\doublespacing (2.0).

\begin{flushright}
	\begin{minipage}{0.6\textwidth} 
		A Dios.
	\end{minipage}
\end{flushright}

\vspace{0.3cm}

\begin{flushleft}
	\begin{minipage}{0.6\textwidth} 
		A mis padres, hermanos...
	\end{minipage}
\end{flushleft}

\vspace{0.3cm}

\begin{flushright}
	\begin{minipage}{0.6\textwidth} 
		A todos los docentes...
	\end{minipage}
\end{flushright}

\vspace{0.3cm}

\begin{flushleft}
	\begin{minipage}{0.6\textwidth} 
		A los amigos...
	\end{minipage}
\end{flushleft}
    \addtocontents{toc}{\protect\setcounter{tocdepth}{-1}}
\tableofcontents 		% Inserta el índice general sin incluirlo en el índice general
\listoftables 			% Inserta el índice de cuadros sin agregarlo al índice general
\listoffigures 			% Inserta el índice de figuras sin agregarlo al índice general
\addtocontents{toc}{\protect\setcounter{tocdepth}{3}} % ¡IMPORTANTE!
    \chapter{Resumen}
\doublespacing % Espaciado doble, puede configurarse con: \singlespacing (1.0), \onehalfspacing (1.5) o\doublespacing (2.0).

Lorem ipsum dolor sit amet, consectetur adipiscing elit. Sed condimentum metus lobortis nulla feugiat, nec gravida dui condimentum. Sed viverra magna eu tortor facilisis imperdiet. Etiam consectetur turpis id convallis dignissim. In facilisis elit sit amet odio vehicula dignissim. Donec eleifend quam id ornare iaculis. Donec vulputate pulvinar nisi, laoreet interdum diam cursus a. Vestibulum non elit suscipit, lacinia purus nec, sodales arcu. In gravida nibh ut velit aliquam, facilisis feugiat orci molestie. Duis sed mauris aliquam, blandit massa sed, molestie nisl. Fusce ornare nisi ante, ut luctus nisl placerat eget. Fusce in efficitur erat, id cursus magna.

\textbf{Palabras clave}: Lorem ipsum dolor...
    \input{5_introduccion.tex}
    
    % MAINMATTER: cambia a numeración arábiga y comienza el contenido de la tesis, se agregan los capítulos incluyendo los archivos con el comando "\input{capitulos/1_xxx.tex}", aquí unos ejemplos. Dentro de cada capítulo, se organizan en jerarquías: Capítulo (mediante \chapter{}, Capítulo 1), Sección (\section{}, 1.1.), Subsección (\subsection{}, 1.1.1.), Subsubsección (\subsubsection{}, negrita).
    \mainmatter
    \chapter{Generalidades}
%%%%%%%%%%%%%%%%%%%%%%%%%%%%%%%%%%%%%%%%%%%%%%%%%%%%%%%%%%%%%%%%%
\doublespacing % Espaciado doble, puede configurarse con: \singlespacing (1.0), \onehalfspacing (1.5) o\doublespacing (2.0).
\raggedbottom % Para evitar el estiramiento o justificado vertical del espacio, sobretodo cuando se tienen muchas imágenes o tablas.
%%%%%%%%%%%%%%%%%%%%%%%%%%%%%%%%%%%%%%%%%%%%%%%%%%%%%%%%%%%%%%%%%
%%%%%%%%%%%%%%%%%%%%%%%%%%%%%%%%%%%%%%%%%%%%%%%%%%%%%%%%%%%%%%%%%
\section{Planteamiento del Problema}

\subsection{Problemática}

Lorem ipsum...

\subsection{Formulación del Problema}

\subsubsection{Problema General}

Lorem ipsum...

\section{Objetivos}

\subsection{Objetivo General}

Lorem ipsum...

\subsection{Objetivos Específicos}

\begin{enumerate}
	\item Lorem ipsum...
	\item Lorem ipsum...
\end{enumerate}

%%%%%%%%%%%%%%%%%%%%%%%%%%%%%%%%%%%%%%%%%%%%%%%%%%%%%%%%%%%%%%%%%
\section{Hipótesis}

\subsection{Hipótesis General}

Lorem ipsum...

\subsection{Hipótesis Específicas}

\begin{enumerate}
	\item Lorem ipsum...
	\item Lorem ipsum...
\end{enumerate}

\section{Justificación}

\subsection{Justificación Social}

Lorem ipsum...

%%%%%%%%%%%%%%%%%%%%%%%%%%%%%%%%%%%%%%%%%%%%%%%%%%%%%%%%%%%%%%%%%
\section{Alcances}

Lorem ipsum...

%%%%%%%%%%%%%%%%%%%%%%%%%%%%%%%%%%%%%%%%%%%%%%%%%%%%%%%%%%%%%%%%%
\section{Limitaciones}

\begin{itemize}
	\item Lorem ipsum...
	
	\item Lorem ipsum...
\end{itemize}

%%%%%%%%%%%%%%%%%%%%%%%%%%%%%%%%%%%%%%%%%%%%%%%%%%%%%%%%%%%%%%%%%
\section{Metodología}

\subsection{Tipo de Investigación}

Lorem ipsum...

\subsection{Población y Muestra}

\subsubsection{Población}

Lorem ipsum....

\subsubsection{Muestra}

Lorem ipsum...

\subsection{Variables e Indicadores}

\subsubsection{Variable Independiente}

\textbf{Lorem ipsum...}

\begin{itemize}
	\item Dimensiones
	\begin{enumerate}
		\item Lorem ipsum...
		\item Lorem ipsum...
	\end{enumerate}
\end{itemize}

\subsubsection{Variables Dependientes}

\textbf{Lorem ipsum...}

\begin{itemize}
	\item Dimensiones
	\begin{enumerate}
		\item Lorem ipsum...
	\end{enumerate}
	
	\item Indicadores
	\begin{enumerate}
		\item Lorem ipsum...
	\end{enumerate}
\end{itemize}

\subsection{Definición Operacional}

\subsubsection{Lorem ipsum...}

\begin{itemize}
	\item Lorem ipsum...
\end{itemize}

\subsection{Técnicas e Instrumentos}

\subsubsection{Técnicas}

\begin{itemize}
	
	\item Lorem ipsum...
	\item Lorem ipsum...	
\end{itemize}

\subsection{Flujo de Trabajo}

Lorem ipsum...
    \chapter{Marco Teórico}
%%%%%%%%%%%%%%%%%%%%%%%%%%%%%%%%%%%%%%%%%%%%%%%%%%%%%%%%%%%%%%%%%
\doublespacing % Espaciado doble, puede configurarse con: \singlespacing (1.0), \onehalfspacing (1.5) o\doublespacing (2.0).
\raggedbottom % Para evitar el estiramiento o justificado vertical del espacio, sobretodo cuando se tienen muchas imágenes o tablas.
%%%%%%%%%%%%%%%%%%%%%%%%%%%%%%%%%%%%%%%%%%%%%%%%%%%%%%%%%%%%%%%%%
%%%%%%%%%%%%%%%%%%%%%%%%%%%%%%%%%%%%%%%%%%%%%%%%%%%%%%%%%%%%%%%%%
\section{Antecedentes}

Lorem ipsum dolor sit amet, consectetur adipiscing elit. Sed condimentum metus lobortis nulla feugiat, nec gravida dui condimentum. Sed viverra magna eu tortor facilisis imperdiet.

\section{Tema 1 del Marco Teórico}

\subsection{Importancia del Tema 1}

Lorem ipsum dolor sit amet, consectetur adipiscing elit. Sed condimentum metus lobortis nulla feugiat, nec gravida dui condimentum. Sed viverra magna eu tortor facilisis imperdiet. Etiam consectetur turpis id convallis dignissim. In facilisis elit sit amet odio vehicula dignissim.

Ejemplo del uso de tablas: Esta distribución se muestra en la Tabla~\ref{tabla:variables_lagos}.

\begin{table}[htbp]
	\centering
	\begin{threeparttable}
		\caption[Variables que caracterizan cuerpos de agua dulce.]{Variables que caracterizan cuerpos de agua dulce.}
		\label{tabla:variables_lagos}
		\begin{tabular}{ m{3.5cm}  m{3.5cm}  m{3.5cm}  m{3.5cm} }
			\hline
			\textbf{Morfometría, hidrología y geografía} & \textbf{Química principal de iones} & \textbf{Disponibilidad de nutrientes} & \textbf{Azar, accidental y biológica} \\
			\hline
			Altitud & Geología local & Geología local & Grado de aislamiento \\
			\hline
			Latitud & Proximidad al mar & Vegetación de la cuenca & Inundaciones y sequías pasadas \\
			\hline
			Área, profundidad y volumen & Clima & Uso de tierra & Clima fluctuante \\
			\hline
			Precipitación, evaporación y tiempo de retención & Sedimentos & Sedimentos & Perturbación por eventos naturales y actividad humana \\
			\hline
		\end{tabular}
		\begin{tablenotes}
			\item Nota: Tabla modificada de (\cite{adrian2009lakes}).
		\end{tablenotes}
	\end{threeparttable}
\end{table}


    \chapter{Ejemplos de Uso}
%%%%%%%%%%%%%%%%%%%%%%%%%%%%%%%%%%%%%%%%%%%%%%%%%%%%%%%%%%%%%%%%%
\doublespacing % Espaciado doble, puede configurarse con: \singlespacing (1.0), \onehalfspacing (1.5) o\doublespacing (2.0).
\raggedbottom % Para evitar el estiramiento o justificado vertical del espacio, sobretodo cuando se tienen muchas imágenes o tablas.
%%%%%%%%%%%%%%%%%%%%%%%%%%%%%%%%%%%%%%%%%%%%%%%%%%%%%%%%%%%%%%%%%
%%%%%%%%%%%%%%%%%%%%%%%%%%%%%%%%%%%%%%%%%%%%%%%%%%%%%%%%%%%%%%%%%

\section{Inclusión de Figuras}

Para incluir una imagen en el flujo de texto usar:

\begin{figure}[htbp]
	\centering
        \caption[Este texto va al índice de figuras, no debe  omitirse y normalmente es el mismo que la descripción.]{Este texto es el pie de figura o descripción.}
        \label{fig:ejemplo}
	\includegraphics[width=0.6\linewidth]{example-image}
\end{figure}

La configuración [htbp] da flexibilidad a \LaTeX para ubicar la imagen de la mejor forma posible.

El orden de los comandos es importante: para que el título de la figura esté arriba debe colocarse primero caption[]\{ \}. Dentro del los corchetes [] va el nombre de la figura que irá al índice de figuras, no colocar nada hará que no aparezca en el índice de figuras. Dentro de las llaves \{ \} va la descripción de la figura y aparece encima de la imagen.

El comando label\{ \} genera una referencia hacia esta figura incluida, se referencia a las figuras dentro del texto mediante este label el comando ref\{\}: ... La Figura \ref{fig:ejemplo}..., el label puede ser cualquier texto, en el ejemplo se tiene "fig:ejemplo".

El comando que incrusta las figuras es includegraphics[]\{ \}. Dentro de los corchetes [] se configura el ancho o alto de la figura con $width=0.6 /linewidth$ donde 0.6 representa la fracción del ancho total de la página que ocupará la imagen. En las llaves \{ \} se llama a la ubicación y nombre del archivo, de acuerdo a la estructura, las figuras deben estar en "figuras/xxxx.png". Se aceptan formatos de imagen png, jpg, jpeg, pdf, entre otros.



\section{Inclusión de Tablas}

Un ejemplo de utilización de tablas, se muestra a continuación:

\begin{table}[htbp]
	\centering
	\begin{threeparttable}
		\caption[Niveles de procesamiento para los datos de Colección 2, Nivel-1.]{Niveles de procesamiento para los datos de Colección 2, Nivel-1.}
		\label{tabla:ejemplo}
		\begin{tabular}{ m{5cm}  m{10cm} }
			\hline
			\textbf{Nivel de Procesamiento} & \textbf{Descripción} \\
			\hline
			Terrain Precision Correction (L1TP) & Calibrado radiométricamente y ortorrectificado usando puntos de control. \\
			\hline
			Systematic Terrain Correction (L1GP) & Calibrado radiométricamente, con correcciones geométricas sistemáticas y modelos de elevación digital para corrección de desplazamiento de relieve. \\
			\hline
			Geometric Systematic Correction (L1GS) & Calibrado radiométricamente solo con correcciones geométricas sistemáticas. \\
			\hline
		\end{tabular}
		\begin{tablenotes}
			\item Nota: Tabla modificada de \parencite{motschmann2021water}.
		\end{tablenotes}
	\end{threeparttable}
\end{table}

La configuración [htbp] permite a \LaTeX tener una flexibilidad al posicionar la tabla, si se desea tener más control de dónde se coloca la imagen, cambiar por [H].

Igual que con las figuras, el orden de los comandos, importa. Primero se coloca caption[]\{ \}. Donde el texto entre corchetes [] es el nombre de la tabla que irá en el Índice de tablas. El texto entre llaves \{ \} es el pie de tabla que aparece encima.

Se necesita una etiqueta lo label para referenciar más adelante a este cuadro, similarmente a las figuras, se las referencia con ref\{ \} ... La Tabla \ref{tabla:ejemplo}...

La forma de escribir tablas es genérica, revisar la plantilla.

Se puede añadir una nota al final de la tabla con el comando begin\{tablenotes\}, como se muestra en el ejemplo.



\section{Inclusión de Fórmulas y Expresiones Matemáticas}

Hay muchas formas de colocar expresiones matemáticas y fórmulas. Para las expresiones matemáticas dentro del texto, usar el símbolo \$ \$ dos veces y dentro el texto matemático . La siguiente línea es un ejemplo:

...estas se definen a través de los 2 extremos de la recta detectada, definidos como $P(x_1, y_1$) y $P(x_2, y_2)$. Por ejemplo, se muestra...

Para escribir una fórmula que debe ser enumerada, escribir las fórmulas o expresiones con el entorno begin\{equation\} y su respectivo label, como:

\begin{equation}
	I_{c}'(x, y) = \max(I_{c}(x, y), 0) \quad \text{para} \quad c = 1, 2, \dots, 6
	\label{ecu:ejemplo}
\end{equation}

Similarmente con las figuras y tablas, también es necesario un label\{ \} que lo identifique y pueda ser referencia. ... La Ecuación \ref{ecu:ejemplo} muestra...

Revisar documentación sobre cómo escribir expresiones y fórmulas matemáticas.

\section{Forma de Citar}

La forma de citar es mediante dos comandos, a saber: \textbf{cite} y \textbf{parencite}, ambos están en formato APA, lo que los diferencia son los paréntesis que parencite genera automáticamente alrededor de la cita en cuestión, por ejemplo:

Lorem ipsum dolor sit amet, consectetur adipiscing elit. Sed condimentum metus lobortis nulla feugiat, nec gravida dui condimentum (\cite{wood2021contemporary}). Aquí se utilizó \textit{cite}, por lo que los paréntesis se tuvieron que escribir \textit{manualmente}.

Lorem ipsum dolor sit amet, consectetur adipiscing elit \parencite{wood2021contemporary}. Aquí se utilizó \textit{parencite}, por lo que los paréntesis se escribieron \textit{automáticamente}.

Para poder citar, las referencias deben estar bien añadidas el archivo bibliografia.bib, ver los ejemplos dentro del archivo respectivo. Las citas pueden ser sacadas desde Google Scholar -> Buscar artículo -> Presionar botón Citar -> BibTex -> Copiar a bibliografia.bib



    % Añadir otros capítulos
    % Finalmente, en los capítulos Conclusiones y Recomendaciones es importante evitar modificarlos por sus configuraciones especiales. A diferencia de otros capítulos convencionales, estos cuando son renderizados no comienzan con CAPÍTULO XXX Conclusiones, sino simplemente "Conclusiones" y centrado, lo mismo para "Recomendaciones".
    \newpage
\addcontentsline{toc}{chapter}{Conclusiones} % Añade "Conclusiones" al índice
\vspace*{2cm} % Ajusta el espaciado vertical si es necesario
\begin{center}
	\Huge \textbf{Conclusiones} % Centra y aplica el formato
\end{center}
\doublespacing % Espaciado doble, puede configurarse con: \singlespacing (1.0), \onehalfspacing (1.5) o\doublespacing (2.0).
\raggedbottom % Para evitar el estiramiento o justificado vertical del espacio, sobretodo cuando se tienen muchas imágenes o tablas.
%%%%%%%%%%%%%%%%%%%%%%%%%%%%%%%%%%%%%%%%%%%%%%%%%%%%%%%%%%%%%%%%%
%%%%%%%%%%%%%%%%%%%%%%%%%%%%%%%%%%%%%%%%%%%%%%%%%%%%%%%%%%%%%%%%%
\begin{enumerate}
	\item Lorem ipsum dolor sit amet, consectetur adipiscing elit.
	
	\item Lorem ipsum dolor sit amet, consectetur adipiscing elit.
	
	\item Lorem ipsum dolor sit amet, consectetur adipiscing elit.
    
	\item Lorem ipsum dolor sit amet, consectetur adipiscing elit.
	
	\item Lorem ipsum dolor sit amet, consectetur adipiscing elit.
	
\end{enumerate} % ¡IMPORTANTE!
    \newpage
\addcontentsline{toc}{chapter}{Recomendaciones} % Añade "Recomendaciones" al índice
\vspace*{2cm} % Ajusta el espaciado vertical si es necesario
\begin{center}
	\Huge \textbf{Recomendaciones} % Centra y aplica el formato
\end{center}
\doublespacing % Espaciado doble, puede configurarse con: \singlespacing (1.0), \onehalfspacing (1.5) o\doublespacing (2.0).
\raggedbottom % Para evitar el estiramiento o justificado vertical del espacio, sobretodo cuando se tienen muchas imágenes o tablas.
%%%%%%%%%%%%%%%%%%%%%%%%%%%%%%%%%%%%%%%%%%%%%%%%%%%%%%%%%%%%%%%%%
%%%%%%%%%%%%%%%%%%%%%%%%%%%%%%%%%%%%%%%%%%%%%%%%%%%%%%%%%%%%%%%%%
\begin{itemize}
	\item Lorem ipsum dolor sit amet, consectetur adipiscing elit.
	
	\item Lorem ipsum dolor sit amet, consectetur adipiscing elit.
	
	\item Lorem ipsum dolor sit amet, consectetur adipiscing elit.
    
	\item Lorem ipsum dolor sit amet, consectetur adipiscing elit.
	
	\item Lorem ipsum dolor sit amet, consectetur adipiscing elit.
	
\end{itemize} % ¡IMPORTANTE!

%%%%%%%%%%%%%%%%%%%%%%%%%%%%%%%%%%%%%%%%%%%%%%%%%%%%%%%%%%%%%%%%%
    % Se llama a "bibliografia.tex" que renderiza la bibliografía y la añade al índice, no modificar o evitar modificar esta línea
    \printbibliography % Renderiza la bibligorafía

\addcontentsline{toc}{chapter}{Bibliografía} % Añade la bibliografía al índice como un capítulo % ¡IMPORTANTE!
    
    % Este es el apartado de Apéndices, se pueden añadir archivos de apéndice a partir de aquí
    \appendix
    \chapter{Código de ejemplo XXX}\label{apendice_1}
\singlespacing

Aquí podría ir un código...

\end{document}
