%%%%%%%%%%%%%%%%%%%%%%%%%%%%%%%%%%%%%%%%%%%%%%%%%%%%%%%%%%%%%%%
% PAQUETES PRINCIPALES %%%%%%%%%%%%%%%%%%%%%%%%%%%%%%%%%%%%%%%%
\usepackage[utf8]{inputenc} % Evita problemas con caracteres especiales
\usepackage[T1]{fontenc} % Permite manejo caracteres acentuados y caracteres especiales
\usepackage[spanish]{babel} % Gestiona la traducción de elementos del archivo
\usepackage{tocbibind} % Agrega lista de figuras, tablas y bibliografía al índice
\usepackage{xcolor} % Para comando \textcolor{color}{text}
\usepackage{float}
%%%%%%%%%%%%%%%%%%%%%%%%%%%%%%%%%%%%%%%%%%%%%%%%%%%%%%%%%%%%%%%
% PAQUETE PARA MANEJO E INCLUSIÓN DE IMÁGENES %%%%%%%%%%%%%%%%%
\usepackage{graphicx}
%%%%%%%%%%%%%%%%%%%%%%%%%%%%%%%%%%%%%%%%%%%%%%%%%%%%%%%%%%%%%%%
% PAQUETES DE CONFIGURACIÓN DE MÁRGENES E INTERLINEADO %%%%%%%%
\usepackage{geometry} % Para personalización de márgenes
\geometry{
	left=2.54cm,
	right=2.54cm,
	top=2.54cm,
	bottom=2.54cm
}
\usepackage{setspace} % Gestión de interlineado, opciones: \singlespacing, \onehalfspacing o\doublespacing.
%%%%%%%%%%%%%%%%%%%%%%%%%%%%%%%%%%%%%%%%%%%%%%%%%%%%%%%%%%%%%%%
% CONFIGURACIÓN DE PÁGINAS %%%%%%%%%%%%%%%%%%%%%%%%%%%%%%%%%%%%
\usepackage{pdflscape} % Configuración de página para manejar gráficos o tablas amplias (para girar las páginas horizontalmente)
%%%%%%%%%%%%%%%%%%%%%%%%%%%%%%%%%%%%%%%%%%%%%%%%%%%%%%%%%%%%%%%
% PAQUETES Y CONFIGURACIONES DE TÍTULOS, PIE DE FIGURAS Y TABLAS
\usepackage{titlesec} % Permite la personalización de títulos y secciones
\usepackage{caption} % Permite la personalización de títulos de figuras y tablas
\usepackage{subcaption} % Permite la inclusión de subtítulos en una figura principal
% Si no se tienen tablas ni figuras aún, los comandos \captionsetuo darán mensaje de advertencia, los cuales desaparecerán cuando se coloquen figuras y tablas en el cuerpo
\captionsetup[figure]{justification=RaggedRight,labelsep=newline,labelfont=bf, textfont=it, position=top,singlelinecheck=false}
\captionsetup[table]{justification=RaggedRight,labelsep=newline,labelfont=bf,textfont=it,position=top,singlelinecheck=false}
% Configuración de índice de Tablas
\addto\captionsspanish{\renewcommand{\listtablename}{Índice de Tablas}}
% Cómo se describiran las tablas: Tabla, Cuadro, etc. Modificar de acuerdo a su necesidad
\addto\captionsspanish{\renewcommand{\tablename}{Tabla}}
\usepackage{hyperref} % Manejo de hiperreferencias
%%%%%%%%%%%%%%%%%%%%%%%%%%%%%%%%%%%%%%%%%%%%%%%%%%%%%%%%%%%%%%%
% PAQUETES PARA MANEJO DE TABLAS %%%%%%%%%%%%%%%%%%%%%%%%%%%%%%
\usepackage{array} % Permite la gestión de tablas
\usepackage{multirow} % Permite la configuración de multifila
\usepackage{threeparttable} % Tablas con notas
%%%%%%%%%%%%%%%%%%%%%%%%%%%%%%%%%%%%%%%%%%%%%%%%%%%%%%%%%%%%%%%
% PAQUETE DE RENDERIZACIÓN DE CAJAS PARA TEXTO %%%%%%%%%%%%%%%%
\usepackage{tcolorbox} % Cajas de texto personalizadas
%%%%%%%%%%%%%%%%%%%%%%%%%%%%%%%%%%%%%%%%%%%%%%%%%%%%%%%%%%%%%%%
% PAQUETES PARA USO DE MATEMÁTICAS %%%%%%%%%%%%%%%%%%%%%%%%%%%%
\usepackage{amsmath, amssymb}
\usepackage{newtxtext,newtxmath} % Modifica la fuente a una parecida a Times New Roman
%%%%%%%%%%%%%%%%%%%%%%%%%%%%%%%%%%%%%%%%%%%%%%%%%%%%%%%%%%%%%%%
% CONFIGURACIÓN PARA INCUSIÓN DE CÓDIGOS EN EL TEXTO %%%%%%%%%%
\usepackage{listings} % Para código fuente
\usepackage{inconsolata} % Fuente monoespaciada
\lstset{
	language=Python,
	backgroundcolor=\color{white},
	basicstyle=\footnotesize\ttfamily,
	keywordstyle=\color{blue}\bfseries,
	stringstyle=\color{red},
	commentstyle=\color{green!50!black},
	frame=single,
	breaklines=true,
	numbers=left,
	numberstyle=\tiny\color{gray},
	tabsize=4
}
%%%%%%%%%%%%%%%%%%%%%%%%%%%%%%%%%%%%%%%%%%%%%%%%%%%%%%%%%%%%%%%
% CONFIGURACIÓN Y GESTIÓN DE BIBLIOGRAFÍA CON COMPILADOR BIBLATEX
\usepackage[style=apa, backend=biber]{biblatex}
\usepackage[babel]{csquotes} % Manejo avanzado de citas
% Mapeo de bibliografía estilo APA en español
\DeclareLanguageMapping{spanish}{spanish-apa}
\DefineBibliographyExtras{spanish}{
	\renewcommand*{\finalnamedelim}{\addspace y\space}
	\renewcommand*{\multinamedelim}{\addcomma\space}
	\renewcommand*{\multilistdelim}{\addcomma\space}
}
\makeatletter
\DefineBibliographyExtras{spanish}
{
	\setcounter{smartand}{1}
	\let\lbx@finalnamedelim=\lbx@es@smartand
	\let\lbx@finallistdelim=\lbx@es@smartand
}
\makeatother  
% Añadir archivo de bibliografía
\addbibresource{bibliografia.bib}
%%%%%%%%%%%%%%%%%%%%%%%%%%%%%%%%%%%%%%%%%%%%%%%%%%%%%%%%%%%%%%%
% Numeración de páginas %%%%%%%%%%%%%%%%%%%%%%%%%%%%%%%%%%%%%%%
\pagestyle{plain}