\chapter{Marco Teórico}
%%%%%%%%%%%%%%%%%%%%%%%%%%%%%%%%%%%%%%%%%%%%%%%%%%%%%%%%%%%%%%%%%
\doublespacing % Espaciado doble, puede configurarse con: \singlespacing (1.0), \onehalfspacing (1.5) o\doublespacing (2.0).
\raggedbottom % Para evitar el estiramiento o justificado vertical del espacio, sobretodo cuando se tienen muchas imágenes o tablas.
%%%%%%%%%%%%%%%%%%%%%%%%%%%%%%%%%%%%%%%%%%%%%%%%%%%%%%%%%%%%%%%%%
%%%%%%%%%%%%%%%%%%%%%%%%%%%%%%%%%%%%%%%%%%%%%%%%%%%%%%%%%%%%%%%%%
\section{Antecedentes}

Lorem ipsum dolor sit amet, consectetur adipiscing elit. Sed condimentum metus lobortis nulla feugiat, nec gravida dui condimentum. Sed viverra magna eu tortor facilisis imperdiet.

\section{Tema 1 del Marco Teórico}

\subsection{Importancia del Tema 1}

Lorem ipsum dolor sit amet, consectetur adipiscing elit. Sed condimentum metus lobortis nulla feugiat, nec gravida dui condimentum. Sed viverra magna eu tortor facilisis imperdiet. Etiam consectetur turpis id convallis dignissim. In facilisis elit sit amet odio vehicula dignissim.

Ejemplo del uso de tablas: Esta distribución se muestra en la Tabla~\ref{tabla:variables_lagos}.

\begin{table}[htbp]
	\centering
	\begin{threeparttable}
		\caption[Variables que caracterizan cuerpos de agua dulce.]{Variables que caracterizan cuerpos de agua dulce.}
		\label{tabla:variables_lagos}
		\begin{tabular}{ m{3.5cm}  m{3.5cm}  m{3.5cm}  m{3.5cm} }
			\hline
			\textbf{Morfometría, hidrología y geografía} & \textbf{Química principal de iones} & \textbf{Disponibilidad de nutrientes} & \textbf{Azar, accidental y biológica} \\
			\hline
			Altitud & Geología local & Geología local & Grado de aislamiento \\
			\hline
			Latitud & Proximidad al mar & Vegetación de la cuenca & Inundaciones y sequías pasadas \\
			\hline
			Área, profundidad y volumen & Clima & Uso de tierra & Clima fluctuante \\
			\hline
			Precipitación, evaporación y tiempo de retención & Sedimentos & Sedimentos & Perturbación por eventos naturales y actividad humana \\
			\hline
		\end{tabular}
		\begin{tablenotes}
			\item Nota: Tabla modificada de (\cite{adrian2009lakes}).
		\end{tablenotes}
	\end{threeparttable}
\end{table}

